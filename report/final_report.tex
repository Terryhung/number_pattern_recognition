% --------------------------------------------------------------
% This is all preamble stuff that you don't have to worry about.
% Head down to where it says "Start here"
% --------------------------------------------------------------

\documentclass[12pt]{article}

\usepackage{fontspec}
\usepackage{xeCJK}
\usepackage[margin=1in]{geometry}
\usepackage{amsmath,amsthm,amssymb}
\usepackage{graphicx}

\setCJKmainfont{LiHei Pro}
\XeTeXlinebreaklocale zh
\XeTeXlinebreakskip = 0pt plus 1pt

% ------ Thm. Def. etc. ---------

% Theorem Styles
\newtheorem{theorem}{Theorem}[section]
\newtheorem{lemma}[theorem]{Lemma}
\newtheorem{proposition}[theorem]{Proposition}
\newtheorem{corollary}[theorem]{Corollary}
% Definition Styles
\theoremstyle{definition}
\newtheorem{definition}{Definition}[section]
\newtheorem{example}{Example}[section]
\theoremstyle{remark}
\newtheorem{remark}{Remark}

% ------ For pasting codes ------
\usepackage{listings}
\usepackage{color}

\definecolor{dkgreen}{rgb}{0,0.6,0}
\definecolor{gray}{rgb}{0.5,0.5,0.5}
\definecolor{mauve}{rgb}{0.58,0,0.82}

\lstset{frame=tb,
  language=C,
  aboveskip=3mm,
  belowskip=3mm,
  showstringspaces=false,
  columns=flexible,
  basicstyle={\small\ttfamily},
  numbers=none,
  numberstyle=\tiny\color{gray},
  keywordstyle=\color{blue},
  commentstyle=\color{dkgreen},
  stringstyle=\color{mauve},
  breaklines=true,
  breakatwhitespace=true,
  tabsize=3
}
% -----------------------------------

\begin{document}

% --------------------------------------------------------------
%                         Start here
% --------------------------------------------------------------

\title{Machine Learning Final Report}
\author{洪仲言 B00201015\\
蔡佳文 B00201025}
\maketitle
\section{Algorithm}
\subsection{Linear Model}
\begin{enumerate}
  \item {\em Logistic Regression\/}
  \item {\em Ridge Regression\/}
  \item {\em Linear SVM\/}
\end{enumerate}
\subsection{NonLinear Model}
\begin{enumerate}
  \item {\em SVM\/}
  \item {\em Random Forest\/}
  \item {\em CNN\/}
\end{enumerate}
\section{Preprocess}
  \subsection{Photo}
    \begin{enumerate}
      \item \large{\em \color{red}Resize\/}\\
        因為我們看見大家寫的字都東倒西歪的,如果直接將抓下來的資料拿來訓練一定很慘烈。\\
        所以我們用了兩種方式進行處理。
        \begin{enumerate}
          \item 將圖片用一個四邊形逼近,以減少太多空白的部分。只留下四邊形後,接著放大成122*105
          \item 將圖片用一個四邊形逼近,以減少太多空白的部分。把四邊形移到圖中心。為什麼會想這樣做呢?
            因為擔心放大後失去字的結構。可能『龍』這個字會因為放大,整團擠在一起。
        \end{enumerate}
      \item \large{\em \color{red}HOG\/}\\
        <<Histograms of Oriented Gradients for Human Detection>>是在2005年CVPR上發表的\\
        想用這個方法的原因是:就如同論文想要找到人在圖片裡麵和其他物體的互動(車子之類的),那他可是用梯度的方法找到
        人和車子。那我們是文字,我們拿到的資料裡面只有字還有一堆空白處,所以我們如果成順利找到字,並且將文字和空白分離
        這樣也許可以解決大家同樣的字在122*105裡面,出現在不同位子的情況。
    \end{enumerate}
  \subsection{Class}
    \begin{enumerate}
      \item \large{\em \color{red}合併\/}\\
        因為在track0上面,一判斷成壹也算是得分,反之亦然。所以我想說可不可以在class上面降維,把所有class > 21的
        都減掉10。但是結果似乎不太理想,可能令model感到疑惑了。
    \end{enumerate}
\section{Bagging and Blending}
  \subsection{Bagging}
    \begin{enumerate}
      \item 對linear svm 進行Bagging 100 參數C: 1 \\
          效果不錯 0.28 $ \to $ 0.267
      \item 對kernel svm 進行Bagging 100  參數$ \text{kernel: rbf, C: 100,} \gamma \text{: 0.1} $\\
          實驗進行到一半\dots\dots 收到網管的信memory leaks QQ ,雖然比賽很重要
          ,但是跟實驗室學長姐的感情也很重要所以忍痛放棄這個實驗。
      \item 對ramdom forest 進行 Bagging 100 參數 870顆樹\\
          Random Forest 本身就是一個Bagging的演算法了,想說試試看會不會發生什麼怪異的事情,但是
          跑了好久還沒跑完,所以在四天後放棄這個探險。
    \end{enumerate}
  \subsection{Blending}
    \begin{enumerate}
      \item 對進行公平投票的方法,看哪個過半數,如果都沒有的話,隨機選一個答案\\
        <{\em SVM:\/} kernel=rbf C=125 $ \gamma = 0.12 $>\\ 
        <{\em Random Forest:\/} tree=870 max\_feature=sqrt> \\
        <{\em SVM linear:\/} c=1 bagging=100>\\
        成果有進步 0.24 (三個臭皮匠勝過一個臭皮匠)
      \item 對進行公平投票的方法,看哪個過半數,如果都沒有的話,選一個較多的,如果有一樣票數最多的,在隨機選一個\\
        <{\em SVM:\/} kernel=rbf C=125 $ \gamma = 0.12 $>\\ 
        <{\em Random Forest:\/} tree=870 max\_feature=sqrt> \\
        <{\em SVM linear:\/} c=1 bagging=100>\\
        <{\em Logistic regression:\/} c=1>\\
        <{\em Ridge regression:\/} c=10>\\
        成果有進步 0.25 但是相較於上一個Blending竟然退步了,可能是因為人多口雜,好的事情被淹沒了
    \end{enumerate}
\end{document}
